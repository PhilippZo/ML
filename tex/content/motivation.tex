%\section{Motivation}
%The last years were heavily defined by the defestating results of the COVID 19 pandmic \cite{covid}
%and thus we were greatly impactaed by that viurs. One of the most dominant factors of every new desease is its growth
%i.e the confirmed cases in a given population. A most common approach to this is relying on the exponential groeth, and thus fitting it this way.
%The main idea of this project is to find an efficient way
%to predict the confirmed cases of COVID 19 patients through a selection of features that might corrolate.
%Since COVID is transmitted though every day contact via aerosols \cite{covid} the features will be a selection of weather data 
%containing temperatures and air pressure. It is the idea to look for dependencies of, for example, good weather that might lead to an increase outgoing 
%of people and thus a higher spread of the virus or the opposite case, where bad weather would weaken the immunn system and thus increase the general test rate with 
%a resulting increas of confirmed cases. A geological feature will not be included to isolate the problme from social status and other political factors.
%To fully understand the impact of the weather it is of great importance to also include its time dependencies since four days of bad waether might have a different impact than an alternating 
%weather cycle. Thus this project will aim to train a recurrent neural network \cite{keras} to satisfy the time sequenced data with the goal to predict the confimred corona cases of several cities.

\section{Motivation}
The last years were heavily defined by the devastating results of the COVID-19 pandemic \cite{covid}, and we were greatly impacted by this virus. One of the most critical factors of any new disease is its growth, i.e., the confirmed cases in a given population. A common approach to analyzing this growth is to rely on exponential models for fitting the data.
The main idea of this project is to develop an efficient way to predict the confirmed cases of COVID-19 patients using a selection of features that might correlate with the spread of the virus. As COVID-19 is primarily transmitted through everyday contact via aerosols \cite{covid}, the selected features will focus on weather data, such as temperatures and air pressure. The goal is to investigate potential dependencies, such as how good weather might lead to an increase in outdoor activities, resulting in a higher spread of the virus. Conversely, bad weather might weaken the immune system, leading to increased testing and subsequent confirmation of cases. To isolate the problem from social status and other political factors, geological features will not be included in the analysis.
To fully understand the impact of weather on COVID-19 cases, it is crucial to consider time dependencies. 
Thus, this project aims to train a recurrent neural network \cite{keras} to handle time-sequenced data and predict the confirmed corona cases of several cities. By incorporating temporal patterns, the model can better capture the dynamics of COVID-19 cases and improve prediction accuracy.
